\documentclass{beamer}
\usetheme{Madrid}
\usepackage[utf8]{inputenc}
\usepackage{physics}
\usepackage{csquotes}
\def\signed #1{{\leavevmode\unskip\nobreak\hfil\penalty50\hskip1em
  \hbox{}\nobreak\hfill #1%
  \parfillskip=0pt \finalhyphendemerits=0 \endgraf}}

\newsavebox\mybox
\newenvironment{aquote}[1]
  {\savebox\mybox{#1}\begin{quote}\openautoquote\hspace*{-.7ex}}
  {\unskip\closeautoquote\vspace*{1mm}\signed{\usebox\mybox}\end{quote}}

%Information to be included in the title page:
\title{Quantum Error Correction}
\author[GGHLZ]{Louis Golowich \and Wenjie Gong \and Ari Hatzimemos\\ Dylan Li \and Dylan Zhou}
\institute[Harvard University]{Physics 160 \\ Harvard University}
\date[Physics 160 Final Project]{Final Project Presentation, 13 May 2020}

\begin{document}

\frame{\titlepage}

\begin{frame}
    \frametitle{Table of Contents}
    \tableofcontents
\end{frame}

\section{Introduction and Review of Quantum Error Correction}
\begin{frame}
\frametitle{Introduction}
    \begin{aquote}{William Blake}
        To be an Error and to be Cast out is part of God's Design.
    \end{aquote}
    \vspace{15mm}
    \begin{itemize}
        \item Noise as a longstanding problem in information processing systems
            \begin{itemize}
                \item \textit{e.g.}, classical computers, modems, CD players, etc.
            \end{itemize}
        \item Noise is still a problem in quantum information
        \item Key idea: to protect a message against noise, \textit{encode} the message by adding redundant information; even if some information is corrupted, redundancy allows us to \textit{decode} and recover the original message
    \end{itemize}
\end{frame}

\begin{frame}
    \frametitle{Project Framework}
    \begin{itemize}
        \item Goals: 
        \begin{itemize}
            \item to implement various quantum error-correcting codes
            \begin{itemize}
                \item we chose the 3-qubit, 9-qubit, 7-qubit codes
            \end{itemize}
            \item to analyze and compare their performances 
            \begin{itemize}
                \item \textit{when are they effective?}
                \item \textit{when should we use error-correcting codes?}
            \end{itemize} 
        \end{itemize}
        \item Tools:
        \begin{itemize}
            \item Python's Qiskit package
            \item IBM's quantum machines
        \end{itemize}
    \end{itemize}
\end{frame}

\section{The 3-Qubit Codes}
\begin{frame}
    \frametitle{3-Qubit Codes: A Review}
    \textbf{Classical Inspiration}
    \begin{itemize}
        \item Encoding by \textit{repetition codes}:
        \begin{align*}
            &0 \rightarrow 000 \\
            &1 \rightarrow 111.
        \end{align*}
        \item Decoding by \textit{majority voting}:
        \begin{align*}
            \textit{Ex.: }\,\, 001 \rightarrow 0.
        \end{align*}
        \item Analysis: Let $p$ be the probability that a bit is flipped. This method fails when 2 or more bits are flipped, which occurs with probability $3p^2(1-p)+p^3$, so the probability of error is $p_e = 3p^2-2p^3$. Then this method is preferred when $p_e < p$, or $p < 1/2$.
    \end{itemize}
\end{frame}

\begin{frame}
    \frametitle{3-Qubit Codes: A Review}
    \textbf{The Quantum Version: 3-Qubit Bit Flip Code}
    \begin{itemize}
        \item Encoding:
        \begin{align*}
            &\ket{0} \rightarrow \ket{0_L} \equiv \ket{000} \\
            &\ket{1} \rightarrow \ket{1_L} \equiv \ket{111}.
        \end{align*}
    \end{itemize}
\end{frame}

\section{The Shor Code}
\begin{frame}
    \frametitle{The Shor Code}
\end{frame}

\section{The 7-Qubit Code}
\begin{frame}
    \frametitle{7-Qubit Code}
\end{frame}

\end{document}