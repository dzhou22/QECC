\documentclass{beamer}
\usetheme{Madrid}
\usepackage[utf8]{inputenc}
\usepackage{physics}
\usepackage{csquotes}
\def\signed #1{{\leavevmode\unskip\nobreak\hfil\penalty50\hskip1em
  \hbox{}\nobreak\hfill #1%
  \parfillskip=0pt \finalhyphendemerits=0 \endgraf}}

\newsavebox\mybox
\newenvironment{aquote}[1]
  {\savebox\mybox{#1}\begin{quote}\openautoquote\hspace*{-.7ex}}
  {\unskip\closeautoquote\vspace*{1mm}\signed{\usebox\mybox}\end{quote}}

\usepackage{qcircuit}

%Information to be included in the title page:
\title{Quantum Error Correction}
\author[GGHLZ]{Louis Golowich \and Wenjie Gong \and Ari Hatzimemos\\ Dylan Li \and Dylan Zhou}
\institute[Harvard University]{Physics 160 \\ Harvard University}
\date[Physics 160 Final Project]{Final Project Presentation, 13 May 2020}

\begin{document}

\frame{\titlepage}

\begin{frame}
    \frametitle{Table of Contents}
    \tableofcontents
\end{frame}

\section{Introduction and Review of Quantum Error Correction}
\begin{frame}
\frametitle{Introduction}
    \begin{aquote}{William Blake}
        To be an Error and to be Cast out is part of God's Design.
    \end{aquote}
    \vspace{15mm}
    \begin{itemize}
        \item Noise as a longstanding problem in information processing systems
            \begin{itemize}
                \item \textit{e.g.}, classical computers, modems, CD players, etc.
                \item  Noise is still a problem in quantum information
            \end{itemize}
        \item Key idea: to protect a message against noise, \textit{encode} the message by adding redundant information; even if some information is corrupted, redundancy allows us to \textit{decode} and recover the original message
    \end{itemize}
\end{frame}

\begin{frame}
    \frametitle{Project Framework}
    \begin{itemize}
        \item Goals: 
        \begin{itemize}
            \item to implement various quantum error-correcting codes
            \begin{itemize}
                \item we chose the 3-qubit, 9-qubit, 7-qubit codes
            \end{itemize}
            \item to analyze and compare their performances 
            \begin{itemize}
                \item \textit{when are they effective?}
                \item \textit{when should we use error-correcting codes?}
            \end{itemize} 
        \end{itemize}
        \item Tools:
        \begin{itemize}
            \item Python's Qiskit package
            \item IBM's quantum machines
        \end{itemize}
    \end{itemize}
\end{frame}

\section{The 3-Qubit Codes}
\begin{frame}
    \frametitle{3-Qubit Codes: A Review}
    \textbf{Classical Inspiration}
    \begin{itemize}
        \item Encoding by \textit{repetition codes}:
        \begin{align*}
            &0 \rightarrow 000 \\
            &1 \rightarrow 111.
        \end{align*}
        \item Decoding by \textit{majority voting}:
        \begin{align*}
            \textit{Ex.: }\,\, 001 \rightarrow 0.
        \end{align*}
        \item Analysis: Let $p$ be the probability that a bit is flipped. This method fails when 2 or more bits are flipped, which occurs with probability $3p^2(1-p)+p^3$, so the probability of error is $p_e = 3p^2-2p^3$. Then this method is preferred when $p_e < p$, or $p < 1/2$.
    \end{itemize}
\end{frame}

\begin{frame}
    \frametitle{3-Qubit Codes: A Review}
    \textbf{The Quantum Version: 3-Qubit Bit Flip Code}
    \begin{itemize}
        \item The goal is to correct bit flip errors.
        \item Encoding:
        \begin{align*}
            &\ket{0} \rightarrow \ket{0_L} \equiv \ket{000} \\
            &\ket{1} \rightarrow \ket{1_L} \equiv \ket{111}.
        \end{align*}
        \item Encoding circuit for 3-qubit bit flip code:
    
        \vspace{5mm}
        \hspace{48mm}
        \Qcircuit @C=1em @R=1.5em {
        \lstick{\ket{\psi}} & \ctrl{1} & \ctrl{2} & \qw \\ 
        \lstick{\ket{0}} & \targ & \qw & \qw \\
        \lstick{\ket{0}} & \qw & \targ & \qw
        }

    \end{itemize}
\end{frame}

\begin{frame}
    \frametitle{3-Qubit Codes: A Review}
    \textbf{The Quantum Version: 3-Qubit Bit Flip Code}
    \begin{itemize}
        \item Suppose there is a bit flip error after encoding:
    
        \vspace{5mm}
        \hspace{37mm}
        \Qcircuit @C=1em @R=1.5em {
        \lstick{\ket{\psi}} & \ctrl{1} & \ctrl{2} & \qw & \multigate{2}{E_{\text{bit}}} & \qw\\ 
        \lstick{\ket{0}} & \targ & \qw & \qw & \ghost{E_{\text{bit}}} & \qw\\
        \lstick{\ket{0}} & \qw & \targ & \qw & \ghost{E_{\text{bit}}} & \qw
        }\vspace{5mm}

        \item Error Detection (or \textit{syndrome diagnosis}):
            \begin{itemize}
                \item we would like to determine which, if any, of the qubits have been corrupted
                \item four error syndromes: no error, bit flip on qubit one, bit flip on qubit two, bit flip on qubit three
            \end{itemize}

    \end{itemize}
\end{frame}

\begin{frame}
    \frametitle{3-Qubit Codes: A Review}
    \textbf{The Quantum Version: 3-Qubit Bit Flip Code}
    \begin{itemize}
        \item Error Detection (or \textit{syndrome diagnosis}):
            \begin{itemize}
                \item we can diagnose the syndrome using two ancillary qubits:
                
            \vspace{5mm}
            \hspace{10mm}
            \Qcircuit @C=1em @R=1.5em {
            \lstick{\ket{\psi}} & \ctrl{1} & \ctrl{2} & \qw & \multigate{2}{E_{\text{bit}}} & \qw & \ctrl{3} & \qw & \ctrl{4} &\qw &\qw\\ 
            \lstick{\ket{0}} & \targ & \qw & \qw & \ghost{E_{\text{bit}}} & \qw & \qw & \ctrl{2} & \qw & \qw\ & \qw\\
            \lstick{\ket{0}} & \qw & \targ & \qw & \ghost{E_{\text{bit}}} & \qw &  \qw & \qw & \qw & \ctrl{2} & \qw\\
            & & & &  &\lstick{\ket{0}} & \targ & \targ & \qw & \qw  & \qw & \meter\\
            & & & &  &\lstick{\ket{0}} & \qw & \qw & \targ & \targ & \qw &\meter
            } \vspace{5mm}

            \item Based on measurement results, we know where the error occured.
            \end{itemize}

    \end{itemize}
\end{frame}

\begin{frame}
    \frametitle{3-Qubit Codes: A Review}
    \textbf{The Quantum Version: 3-Qubit Bit Flip Code}
    \begin{itemize}
        \item Error Correction (or \textit{recovery}):

            \includegraphics[scale=0.35]{3qb-circuit.png}
    \end{itemize}
\end{frame}

\section{The Shor Code}
\begin{frame}
    \frametitle{The Shor Code}
\end{frame}

\section{The 7-Qubit Code}
\begin{frame}
  \frametitle{7-Qubit Code}
  Encodes 1 logical qubit using 7 physical qubits:
  {\tiny
    \begin{align*}
      \ket{\overline{0}} &= \frac{\ket{0000000}+\ket{1010101}+\ket{0110011}+\ket{1100110}+\ket{0001111}+\ket{1011010}+\ket{0111100}+\ket{1101001}}{\sqrt{8}} \\
      \ket{\overline{1}} &= \frac{\ket{1111111}+\ket{0101010}+\ket{1001100}+\ket{0011001}+\ket{1110000}+\ket{0100101}+\ket{1000011}+\ket{0010110}}{\sqrt{8}}
    \end{align*}
    \begin{align*}
      H^{\otimes 7}\ket{\overline{0}} &= \frac{\ket{\overline{0}}+\ket{\overline{1}}}{\sqrt{2}} \\
      H^{\otimes 7}\ket{\overline{1}} &= \frac{\ket{\overline{0}}-\ket{\overline{1}}}{\sqrt{2}}
    \end{align*}
  }
\end{frame}

\begin{frame}
  \frametitle{7-Qubit Code}
  \vspace{-1cm}
  {\tiny
    \begin{align*}
      \ket{\overline{0}} &= \frac{\ket{0000000}+\ket{1010101}+\ket{0110011}+\ket{1100110}+\ket{0001111}+\ket{1011010}+\ket{0111100}+\ket{1101001}}{\sqrt{8}} \\
      \ket{\overline{1}} &= \frac{\ket{1111111}+\ket{0101010}+\ket{1001100}+\ket{0011001}+\ket{1110000}+\ket{0100101}+\ket{1000011}+\ket{0010110}}{\sqrt{8}}
    \end{align*}
    \begin{align*}
      H^{\otimes 7}\ket{\overline{0}} &= \frac{\ket{\overline{0}}+\ket{\overline{1}}}{\sqrt{2}} \\
      H^{\otimes 7}\ket{\overline{1}} &= \frac{\ket{\overline{0}}-\ket{\overline{1}}}{\sqrt{2}}
    \end{align*}
  }
  \begin{itemize}
  \item Of the 16 bit strings above, any two differ by $\geq 3$ bits
  \item Intuition: therefore a single bit flip can be recovered
    \begin{itemize}
    \item $X$ error flips bit in $\ket{\overline{0}},\ket{\overline{1}}$
    \item $Z$ error flips bit in $H^{\otimes 7}\ket{\overline{0}},H^{\otimes 7}\ket{\overline{1}}$
    \end{itemize}
  \end{itemize}
\end{frame}

% \begin{frame}{7-Qubit Code: Intuition}
%   Example: single bit flip on $1010101$ gives $1011101$
%   \begin{center}
%     \begin{tabular}{ccc}
%       & Bit string & Number bits that disagree \\
%       \hline
%       Want to decode: & 1011101 & \\
%       \hline
%       & 101\textcolor{red}{0}101 & 1 \\
%       & \textcolor{red}{0}0\textcolor{red}{0}\textcolor{red}{0}\textcolor{red}{0}0\textcolor{red}{0} & 5 \\
%     \end{tabular}
%   \end{center}
% \end{frame}

\begin{frame}{Example recovery for $X$ error in qubit 3}
  TODO
\end{frame}

\end{document}