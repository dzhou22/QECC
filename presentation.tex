\documentclass{beamer}
\usetheme{Madrid}
\usepackage[utf8]{inputenc}
\usepackage{physics}
\usepackage{csquotes}
\def\signed #1{{\leavevmode\unskip\nobreak\hfil\penalty50\hskip1em
  \hbox{}\nobreak\hfill #1%
  \parfillskip=0pt \finalhyphendemerits=0 \endgraf}}

\newsavebox\mybox
\newenvironment{aquote}[1]
  {\savebox\mybox{#1}\begin{quote}\openautoquote\hspace*{-.7ex}}
  {\unskip\closeautoquote\vspace*{1mm}\signed{\usebox\mybox}\end{quote}}

%Information to be included in the title page:
\title{Quantum Error Correction}
\author[GGHLZ]{Louis Golowich \and Wenjie Gong \and Ari Hatzimemos\\ Dylan Li \and Dylan Zhou}
\institute[Harvard University]{Physics 160 \\ Harvard University}
\date[Physics 160 Final Project]{Final Project Presentation, 13 May 2020}

\begin{document}

\frame{\titlepage}

\begin{frame}
    \frametitle{Table of Contents}
    \tableofcontents
\end{frame}

\section{Introduction and Review of Quantum Error Correction}
\begin{frame}
\frametitle{Introduction}
    \begin{aquote}{William Blake}
        To be an Error and to be Cast out is part of God's Design.
    \end{aquote}
    \vspace{15mm}
    \begin{itemize}
        \item Noise as a longstanding problem in information processing systems
            \begin{itemize}
                \item \textit{e.g.}, classical computers, modems, CD players, etc.
            \end{itemize}
        \item 
        \item Key idea: to protect a message against noise, \textit{encode} the message by adding redundant information; even if some information is corrupted, redundancy allows us to \textit{decode} and recover the original message
    \end{itemize}
\end{frame}

\begin{frame}
    \frametitle{Project Framework}
    \begin{itemize}
        \item Goals: 
        \begin{itemize}
            \item to implement various quantum error-correcting codes (3-qubit, 9-qubit, 7-qubit)
            \item to analyze and compare their performances: \textit{when are they effective?}
        \end{itemize}
        \item Tools:
        \begin{itemize}
            \item Python's Qiskit package
            \item IBM's quantum machines
        \end{itemize}
    \end{itemize}
\end{frame}

\section{The 3-Qubit Codes}
\begin{frame}
    \frametitle{3-qubit codes}
    Classical inspiration: repetition codes and majority voting
\end{frame}

\section{The Shor Code}
\begin{frame}
    \frametitle{The Shor code}
\end{frame}

\section{The 7-Qubit Code}
\begin{frame}
    \frametitle{7-qubit code}
\end{frame}

\end{document}